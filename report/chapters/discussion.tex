%
% The MIT License (MIT)
%
% Copyright (c) 2017 Paul Batty
%
% Permission is hereby granted, free of charge, to any person obtaining a copy
% of this software and associated documentation files (the "Software"), to deal
% in the Software without restriction, including without limitation the rights
% to use, copy, modify, merge, publish, distribute, sublicense, and/or sell
% copies of the Software, and to permit persons to whom the Software is
% furnished to do so, subject to the following conditions:
%
% The above copyright notice and this permission notice shall be included in
% all copies or substantial portions of the Software.
%
% THE SOFTWARE IS PROVIDED "AS IS", WITHOUT WARRANTY OF ANY KIND, EXPRESS OR
% IMPLIED, INCLUDING BUT NOT LIMITED TO THE WARRANTIES OF MERCHANTABILITY,
% FITNESS FOR A PARTICULAR PURPOSE AND NONINFRINGEMENT. IN NO EVENT SHALL THE
% AUTHORS OR COPYRIGHT HOLDERS BE LIABLE FOR ANY CLAIM, DAMAGES OR OTHER
% LIABILITY, WHETHER IN AN ACTION OF CONTRACT, TORT OR OTHERWISE, ARISING FROM,
% OUT OF OR IN CONNECTION WITH THE SOFTWARE OR THE USE OR OTHER DEALINGS IN
% THE SOFTWARE.
%

\section{Discussion}
\label{sec:discussion}

\subsection{Results}

This project started out to find the architecture in and around continuous deployment these where then redefined into three different questions as follows:

\begin{itemize}
  \item Is there a clear or ideal architecture that when creating a new or adapting another system should use or head towards.\\
    \item Where are the common traps and pitfalls found when creating a such a system, interlinked  with goal one what can be changed to avoid them.\\
  \item Does the benefits or creating such a system, in both hours and effort pay off in the end.
\end{itemize}

The first two questions are covered through all the chapters in this paper, both continuous deployment and its variations, seeing that they involve a lot of different technologies and process that deviate from some of the standard practises today. However, as explored in the previous two chapters looking at different ideas and designs being put into place there some are emerging patterns that seem to be gravitated towards.
\\\\
Starting with the stages in the pipeline and how they connect to one and another, starting with the VCS in to the build and then tests. With minor variations for different project and modules using a more modular approach. This system applies to all forms of projects from web applications to embedded systems.
\\\\
Then looking towards the workflow, there seems to be a set idea on using the feature based workflow with a single main branch where all development is carried out. This is achieved via pull request or branch based processes.
\\\\
Moving onwards, the hardware and server set-up. The general principles allow the servers to accomplish every task with easy to scale up and down using the three tired architecture.
\\\\
The final technical section the paper covered what the deployment strategies will depend on what is being deployed and what it is being deployed to, including the general procedures around that. The general idea is to make it work everywhere as quickly as possible with no painful experiences for the end user.
\\\\
As for the second question, while they are not directly mentioned, they can be assumed from the context in which the user is trying to build a continuous deployment system. Such as trying to build a multi-tier server set-up for a small 50 line script. 
\\\\
Alternatively a system with a lot of flaky tests, or not enough tests of different types may struggle during the integration stages of the project. Most of these can be avoided by sticking to the core principles outlined in the paper. The cost of such a system is in the education and training of users on the new workflow and system to make sure that is used effectively.
\\\\
For the last question, the paper took a quick look at the cost benefits and who should bother with such a system in the first place. Rather than looking at the places where it should be used identifying three main factors to take into account when deciding team size, project size and project type. For example a solo developer on a small script can survive without whereas a team of five hundred on  a million plus lines of code probably cannot.

\subsection{Future work}

Judging from the current status of the technologies seen throughout this paper, this area seems to be going along the right tracks, making and automating the process as much as possible with quick feedback for all, in turn making the experience easy for users and developers alike.
\\\\
There is however one area that is lacking, and could be explored further which is embedded and other low level systems, as destroying hardware every build to test can get expensive fast. Similarly creating a simulator for the technology is a whole different project by itself. 
\\\\
Web, desktop and mobile applications are all on the right course with easy to reproduce environments getting easier with Docker and other technologies however embedded and other lower level systems are starting to feel a little on the side lines, therefore this needs more looking into. 
\\\\
Such as how to reproduce the environments without attaching such a huge cost to them. Better integration of the tools into the current systems. In general trying to bring the rich ecosystem surrounding the other development areas in this field.