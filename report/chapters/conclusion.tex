%
% The MIT License (MIT)
%
% Copyright (c) 2017 Paul Batty
%
% Permission is hereby granted, free of charge, to any person obtaining a copy
% of this software and associated documentation files (the "Software"), to deal
% in the Software without restriction, including without limitation the rights
% to use, copy, modify, merge, publish, distribute, sublicense, and/or sell
% copies of the Software, and to permit persons to whom the Software is
% furnished to do so, subject to the following conditions:
%
% The above copyright notice and this permission notice shall be included in
% all copies or substantial portions of the Software.
%
% THE SOFTWARE IS PROVIDED "AS IS", WITHOUT WARRANTY OF ANY KIND, EXPRESS OR
% IMPLIED, INCLUDING BUT NOT LIMITED TO THE WARRANTIES OF MERCHANTABILITY,
% FITNESS FOR A PARTICULAR PURPOSE AND NONINFRINGEMENT. IN NO EVENT SHALL THE
% AUTHORS OR COPYRIGHT HOLDERS BE LIABLE FOR ANY CLAIM, DAMAGES OR OTHER
% LIABILITY, WHETHER IN AN ACTION OF CONTRACT, TORT OR OTHERWISE, ARISING FROM,
% OUT OF OR IN CONNECTION WITH THE SOFTWARE OR THE USE OR OTHER DEALINGS IN
% THE SOFTWARE.
%

\section{Conclusion}
\label{sec:conclusion}

At the start of the paper, several goals where defined. The first to look at and understand the terminology, history and meaning behind the various forms of continuous integration, continuous deployment and automatic build and the surround culture of devops. This was achieved in the first chapter of this paper.
\\\\
The second was to look at examples and other implementations of such system in the real world, taking a look at what went right, what they have in common and deriving the ideal systems from theses studies. This was achieved in the second and third chapters.
\\\\
Thirdly, to look at the cost benefits and who should use such as system in the first place, while looking at the status of the current situation around devops and where it is heading, making sure that it is on track. This took on the role looking at what is missing from the toolset and where thing have been done right.
\\\\
Overall, this project has been a success. The aims set out at the start have been achieved. There are a few minor issues that could be improved upon such as using more case studies to create a better foundation on which to draw from. 
\\\\
In the future, more time should be spend towards looking at creating easily reproducible and testing environments for embedded systems as they are non-existant when compared to the status of web, desktop and mobile development.
\\\\
But, this has been an enjoyable and successfully project and having learnt a great deal about the various technology stacks such as Docker, Vagrant and some of the more unique style of application such as serverless servers. In addition to a guide of which can be used when needed to implement a variation of  such a system