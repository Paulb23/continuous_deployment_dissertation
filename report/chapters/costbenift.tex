%
% The MIT License (MIT)
%
% Copyright (c) 2017 Paul Batty
%
% Permission is hereby granted, free of charge, to any person obtaining a copy
% of this software and associated documentation files (the "Software"), to deal
% in the Software without restriction, including without limitation the rights
% to use, copy, modify, merge, publish, distribute, sublicense, and/or sell
% copies of the Software, and to permit persons to whom the Software is
% furnished to do so, subject to the following conditions:
%
% The above copyright notice and this permission notice shall be included in
% all copies or substantial portions of the Software.
%
% THE SOFTWARE IS PROVIDED "AS IS", WITHOUT WARRANTY OF ANY KIND, EXPRESS OR
% IMPLIED, INCLUDING BUT NOT LIMITED TO THE WARRANTIES OF MERCHANTABILITY,
% FITNESS FOR A PARTICULAR PURPOSE AND NONINFRINGEMENT. IN NO EVENT SHALL THE
% AUTHORS OR COPYRIGHT HOLDERS BE LIABLE FOR ANY CLAIM, DAMAGES OR OTHER
% LIABILITY, WHETHER IN AN ACTION OF CONTRACT, TORT OR OTHERWISE, ARISING FROM,
% OUT OF OR IN CONNECTION WITH THE SOFTWARE OR THE USE OR OTHER DEALINGS IN
% THE SOFTWARE.
%

\section{Cost benefits}
\label{sec:costbenefits}

After looking at the entire pipeline this next section is dedicated to who should use such a system, as in when does it become beneficial to set up all the overhead required to actually bring benefits.

\subsection{Benefits}

There have been a few benefits mentioned throughout the paper the main one being the quick feedback loop that can be provided however that are a lot more that have not yet been mentioned.
\\\\
The first, creating an environment where it works on my machine is no longer an answer as the system is designed to be set-up that the software is tested on all system. This in turn makes the excuse invalid as it has been tests developers can have confidence that the software will work on any system.
\\\\
Secondly, it keeps the number of bugs low and the code quality to a set standard,  as the entire system is automated the tests that are run ensure that a level of confidence that there are no major bugs in the system. If there are that they can be fixed quickly and put out to the customers. The code quality while not a substitute for code reviews can still automatically check the style of the code to ensure that it is consistence. 
\\\\
Thirdly, with the quick feedback loop, there are benefits for everyone not only developers working on the software. As the automation tool is keeps log and can send notification everyone who want to keep in the know or gather information about the software can do so easily. Not just the information about the process but easy access to the binaries / packages to make addition deployment or set-ups.
\\\\
Overall projects with continuous deployment or continuous integration end up with a high base level of quality and bug than those without, as it does not fix or prevent bug from being created it allow developer and manager to see what does, does not and what is needed to make the system work, while preventing old bugs from being introduced to the system. And the feedback in invaluable to everyone involved in the software pipeline, from developers to managers, creating an environment where everyone is more confident in the product as there is log and data available to back the claims ups.


\subsection{Cost}

While the benefits are great there are some greats costs that come with implementing continuous integration, delivery or other forms of these systems.
\\\\


\subsection{Who}

From the lists above it should be clear who can benefit from creating a system, the solo developer or the massive multilevel corporation.

\subsection{Cost benefits final thoughts}