%
% The MIT License (MIT)
%
% Copyright (c) 2017 Paul Batty
%
% Permission is hereby granted, free of charge, to any person obtaining a copy
% of this software and associated documentation files (the "Software"), to deal
% in the Software without restriction, including without limitation the rights
% to use, copy, modify, merge, publish, distribute, sublicense, and/or sell
% copies of the Software, and to permit persons to whom the Software is
% furnished to do so, subject to the following conditions:
%
% The above copyright notice and this permission notice shall be included in
% all copies or substantial portions of the Software.
%
% THE SOFTWARE IS PROVIDED "AS IS", WITHOUT WARRANTY OF ANY KIND, EXPRESS OR
% IMPLIED, INCLUDING BUT NOT LIMITED TO THE WARRANTIES OF MERCHANTABILITY,
% FITNESS FOR A PARTICULAR PURPOSE AND NONINFRINGEMENT. IN NO EVENT SHALL THE
% AUTHORS OR COPYRIGHT HOLDERS BE LIABLE FOR ANY CLAIM, DAMAGES OR OTHER
% LIABILITY, WHETHER IN AN ACTION OF CONTRACT, TORT OR OTHERWISE, ARISING FROM,
% OUT OF OR IN CONNECTION WITH THE SOFTWARE OR THE USE OR OTHER DEALINGS IN
% THE SOFTWARE.
%

\section{Study requirements}

From the now understood concepts and ideas presented, the paper will now shift onto the collection of data and compacting them into the final results. 
\\\\
Before stating what data is used and how it is collected some limitation will be placed on the project. Firstly, the data will be limited to that of continuous deployment and continuous integration. With regards to continuous integration, the data will be looked at as a continuous deployment point of view, as for the most part the difference is minuscule.
\\\\
Secondly while the tools used are important the main focus will be on the architecture of the system rather than what they are using, as this is the main focus of the paper.

\subsection{Goals}

Before any data collection could take place several questions were outlined to form the foundations of this paper and to define a clear goal to head towards, they as follows:

\begin{itemize}
  \item Is there a clear or ideal architecture that when creating a new or adapting another system should use or head towards.\\
    \item Where are the common traps and pitfalls found when creating a such a system, interlinked  with goal one what can be changed to avoid them.\\
  \item Does the benefits or creating such a system, in both hours and effort pay off in the end. \\
\end{itemize}

\subsection{Data Collection}

The data collection process can be split into three distinct parts, the first one containing blogs and articles written by others who have implemented, used or worked with continuous deployment systems. This type of information will make up the majority of the data collected as it is intend to collect real world experience of such systems in order to find the flaws and how they were corrected.
\\\\
The second type of of data will be in the form of other research carried out, due to the abjectly questionable nature about the architecture of a perfect system, the majority of papers are based around quantifying the cost-benefits of the systems or how they work in certain areas rather than how they are put together. Therefore it will be a smaller percentage of the collected data.
\\\\
The third type of data will be white papers and papers written by companies, and as such will have a bias to them, however, they will be used in conjunction with the rest in order to offset said bias as much as possible.
\\\\
The rest of the paper from here onwards will be the results gathered from the data collected in order to answer the question above. The following first section will be aimed at the first two goals, with the second part focusing on the final question.