%
% The MIT License (MIT)
%
% Copyright (c) 2017 Paul Batty
%
% Permission is hereby granted, free of charge, to any person obtaining a copy
% of this software and associated documentation files (the "Software"), to deal
% in the Software without restriction, including without limitation the rights
% to use, copy, modify, merge, publish, distribute, sublicense, and/or sell
% copies of the Software, and to permit persons to whom the Software is
% furnished to do so, subject to the following conditions:
%
% The above copyright notice and this permission notice shall be included in
% all copies or substantial portions of the Software.
%
% THE SOFTWARE IS PROVIDED "AS IS", WITHOUT WARRANTY OF ANY KIND, EXPRESS OR
% IMPLIED, INCLUDING BUT NOT LIMITED TO THE WARRANTIES OF MERCHANTABILITY,
% FITNESS FOR A PARTICULAR PURPOSE AND NONINFRINGEMENT. IN NO EVENT SHALL THE
% AUTHORS OR COPYRIGHT HOLDERS BE LIABLE FOR ANY CLAIM, DAMAGES OR OTHER
% LIABILITY, WHETHER IN AN ACTION OF CONTRACT, TORT OR OTHERWISE, ARISING FROM,
% OUT OF OR IN CONNECTION WITH THE SOFTWARE OR THE USE OR OTHER DEALINGS IN
% THE SOFTWARE.
%

\section*{Introduction}
\label{sec:introduction}

Continuous deployment in no new idea as the field of computer science strives to automated all that it can. However, the idea and implementation of automating the development pipeline, going under the name of continuous integration and continuous deployment is a ever changing field. This has lead to the rise of devops also known as development operations dedicated to maintaining and building such systems with more of a focus on the collaboration of the different team involved in the pipeline such as quality assurance, development and management.
\\\\
During a recent project tasked with the aim to create the pipeline from the ground up there were many question that had to be answered. Such as what is the difference between continuous integration and continuous deployment. How does each part of the system fit into the big picture. What is the end game of the system. This is covered in the first section of this paper.
\\\\
Besides just understanding the reasoning and terms used behind the names, it was time to design and implement a system for the project. In addition to looking at the project that this idea spawned from, the paper will look at other attempts and experience from both industry and academic to understand the ideal architecture. This is covered in the second and third chapters.
\\\\
From here the paper will look at the how this architecture will work in practise, the patterns and the direction that this field is heading towards making sure that it is on the right track. This is covered in the final sections four and five.
\\\\
During the creation of the project there seemed to be no coherent or common meanings behind some of the terms. There also was a lack of architectural designs and implementations with most arguing about the systems cost-benefits.
\\\\
Therefore, the main aim of this paper is help define and understand the systems and terms used in this field. In addition to looking at the common themes used in practise to provide a guide for the creation of other automated systems, and those looking to understand this area more in depth.
