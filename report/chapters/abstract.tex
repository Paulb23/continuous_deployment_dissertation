%
% The MIT License (MIT)
%
% Copyright (c) 2017 Paul Batty
%
% Permission is hereby granted, free of charge, to any person obtaining a copy
% of this software and associated documentation files (the "Software"), to deal
% in the Software without restriction, including without limitation the rights
% to use, copy, modify, merge, publish, distribute, sublicense, and/or sell
% copies of the Software, and to permit persons to whom the Software is
% furnished to do so, subject to the following conditions:
%
% The above copyright notice and this permission notice shall be included in
% all copies or substantial portions of the Software.
%
% THE SOFTWARE IS PROVIDED "AS IS", WITHOUT WARRANTY OF ANY KIND, EXPRESS OR
% IMPLIED, INCLUDING BUT NOT LIMITED TO THE WARRANTIES OF MERCHANTABILITY,
% FITNESS FOR A PARTICULAR PURPOSE AND NONINFRINGEMENT. IN NO EVENT SHALL THE
% AUTHORS OR COPYRIGHT HOLDERS BE LIABLE FOR ANY CLAIM, DAMAGES OR OTHER
% LIABILITY, WHETHER IN AN ACTION OF CONTRACT, TORT OR OTHERWISE, ARISING FROM,
% OUT OF OR IN CONNECTION WITH THE SOFTWARE OR THE USE OR OTHER DEALINGS IN
% THE SOFTWARE.
%

\section*{\centering Abstract}

This paper looks into continuous deployment with a focus on the architecture of such a system from the ground up. Starting with where it came from, what it is and the technologies used. Then looking at what an ideal continuous deployment system would be all the way from developer to user. With a quick look at the cost-benefits and who would benefit from such a system. Then taking a look at what is missing, seeing that in general the tools and technology used is on the right track and destined for success. However, the embedded systems and continuous deployment seem to be out of sync therefore more tools need to be created to aid this area into being brought up to date with rest.

\vspace{1.5cm}

\textbf{Keywords:} Continuous deployment, Continuous integration, Automated builds, agile, VCS, workflow, Extreme programming 